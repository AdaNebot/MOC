\section{Suggested algorithms for basic operations}
Mapping a MOC to a unique sorted list of cells at the deepest
resolution order allows usage of very easy and very fast
algorithms. Basic operations such as unions or intersections can be
computed via bit shifts and simple dichotomic algorithms on sorted
lists.  To reduce as much as possible the memory requirement, a good
practice is to store range sets of continuous cells [minValue
  .. maxValue[, instead of individual cells.

\subsection{Union: moc1 $\cup$ moc2}
\begin{lstlisting}
   Map moc1 to rangeList
   Map moc2 in the same rangeList
   Unmap the resulting rangeList
\end{lstlisting}

\subsection{Intersection: moc1 $\cap$ moc2}
\begin{lstlisting}
   Map moc1 in rangeList1
   foreach order/index of moc2
        shift=2*(maxOrder-order)
        append in a rangeList2 the intersection between
              [index << shift .. index+1 << shift[
        and   the corresponding range(s) of rangeList1
   Unmap rangeList2
\end{lstlisting}

\subsection{Map: moc To rangeList}
\begin{lstlisting}
   foreach order/index of moc
         shift=2*(maxOrder-order)
         append in rangeList [index << shift , (index+1) << shift[
         (the range overlapping must be adjusted)
\end{lstlisting}

\subsection{Unmap: rangeList To moc}
\begin{lstlisting}
   for order = 0 to maxOrder
      end if rangeList is empty
      shift = 2*(maxOrder-order)
      offset = (1<<shift) -1
      foreach range [min..max[ of rangeList
         append in moc order/index where index is in [m1 .. m2[
               m1 = (min+offset) >> shift
               m2 = max >> shift
         remove from rangeList [m1<<shift .. m2<<shift[
\end{lstlisting}

\section{Basic HEALPix functions}
For generating space MOC from observations, or drawing them on the
sphere, an HEALPix library is required. The basic functions available
in all HEALPix libraries are the following :
\begin{itemize}
   \item npix <= coordToNpix(order, alpha,delta) : returns the HEALPix
     cell index containing the alpha,delta coordinates.
   \item ArrayOfNpix <= queryDisc(order, alpha,delta,radius) : returns
     the list of cell indices covering the (long,lat,radius) cone
   \item ArrayOfNpix <= queryPolygon(order, alpha1,delta1, $\cdots$
     alphaN,deltaN): returns the list of cell indices covering the
     spherical polygon
   \item (alpha,delta) <= NpixToCoord(order,npix) : returns the
     coordinates of the center of order/npix cell.
   \item ArrayOf(alpha,delta) <= NpixToCorners(order,npix) : returns
     the corner coordinates of order/npix cell.
\end{itemize}

\section{Basic time functions} 
For generating TMOC from observations, a time library might be required to convert the dates are not expressed in JD TCB
\begin{itemize} 
\item \url{http://www.iausofa.org/}
\item \url{http://javastro.github.io/jsofa/}
\item \url{https://docs.astropy.org/en/stable/time/#module-astropy.time}
\item Obtain the index from the JD: 
\begin{lstlisting}[language=C]
static public long getMicrosec(double jd, long offset) {
    long micron = (long)(jd*DAYMICROSEC);
    return micron + (offset*86400000000L);
}
\end{lstlisting}
\end{itemize} 
